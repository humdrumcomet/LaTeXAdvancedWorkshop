\documentclass{amsart}
%%% Animations
%%%%%%%%%%%%%%%%%%%%%%%%%%%%%%%%%%%%%%%%%%%%%%%%%%%%%%%%%%%%%%%%%%%%%%%%%%%%%%%%%%%%%%%%

%%<*n2BeamAnim>
    %\begin{figure}[htbp]
        %\begin{centering}
            %\animategraphics[controls, buttonsize=1.0em, width=\textwidth]{10}{example-image-}{1}{240}
            %\caption{Video of \ch{N2} plasma and beam forming}
            %\label{fig:n2PlasBeamVid}
        %\end{centering}
    %\end{figure}
%%</n2BeamAnim>
%<*lorenzAnimate>
    \begin{figure}[htbp]
        \begin{centering}
            \includestandalone[width=0.85\textwidth]{\assetPath/Images/Tikz/lorenz/lorenzTikzAnim}
            \caption{A tikz sketch}
            \label{fig:animeLorenz}
        \end{centering}
    \end{figure}
%</lorenzAnimate>
%<*golAnimate>
    \begin{figure}[htbp]
        \begin{centering}
            %% arara: lualatex: { shell: true}
% arara: lualatex: { shell: true}

\providecommand{\importPath}{../../../../Shared/Imports}
\providecommand{\assetPath}{../../../../Assets}
\providecommand{\codePath}{../../../../Shared/LuaCalcs}
\providecommand{\sharedPath}{../../../../Shared}
\providecommand{\plotCodePath}{../../../../Assets}

\documentclass[tikz, crop=false, export]{standalone}

\input{\codePath/luaDatSetup}
\usepackage[mode=buildnew, subpreambles=false]{standalone}
\usepackage{import}
%\usepackage[a-1b]{pdfx}
\usepackage{indentfirst}
\usepackage[doublespacing]{setspace}
\usepackage[left=1.5in, right=1in, top=1in, bottom=1in]{geometry}
\usepackage{xurl}
\usepackage{hyperref}
\usepackage[tbtags]{amsmath}
\usepackage{amsfonts}
\usepackage{amssymb}
\usepackage{fancyhdr}
\usepackage[T1]{fontenc}
\usepackage{comment}
\usepackage{xparse}
\usepackage{xcolor}
\usepackage{colortbl}
\usepackage{siunitx}
\usepackage{chemformula}
\usepackage{ifthen}
\usepackage[gobble=auto]{pythontex}
\usepackage{catchfilebetweentags}
\usepackage[style=ieee,backend=biber]{biblatex}
\usepackage{csvsimple}
\usepackage{longtable}
\usepackage{makecell}
\usepackage{multicol}
\usepackage{multirow}
\usepackage{float}
\usepackage{enumitem}
\usepackage{graphicx}

\usepackage{tikz}
    \usetikzlibrary{math, arrows, circuits.ee.IEC, positioning, shapes.arrows, shapes.geometric, external}
\usepackage{pgfplots}
    \pgfplotsset{compat=newest, compat/show suggested version=false}
    \usepgfplotslibrary{groupplots}
\usepackage[siunitx,american voltages, american currents, RPvoltages]{circuitikz}
\usepackage{tikzscale}

\usepackage{animate}
\usepackage{caption}
\usepackage{subcaption}
\usepackage{listings}
\usepackage{chngcntr}
\usepackage[titletoc]{appendix}
\usepackage[titles]{tocloft}
\usepackage{xpatch}
\usepackage[debug, toc, section=section, acronym, symbols]{glossaries} % Glossaries package

%Get rounded wire jumps
\tikzset{
    declare function={% in case of CVS which switches the arguments of atan2
        atan3(\a,\b)=ifthenelse(atan2(0,1)==90, atan2(\a,\b), atan2(\b,\a));},
        kinky cross radius/.initial=+.125cm,
        @kinky cross/.initial=+, kinky crosses/.is choice,
        kinky crosses/left/.style={@kinky cross=-},kinky crosses/right/.style={@kinky cross=+},
        kinky cross/.style args={(#1)--(#2)}{
        to path={
          let \p{@kc@}=($(\tikztotarget)-(\tikztostart)$),
              \n{@kc@}={atan3(\p{@kc@})+180} in
          -- ($(intersection of \tikztostart--{\tikztotarget} and #1--#2)!%
                 \pgfkeysvalueof{/tikz/kinky cross radius}!(\tikztostart)$)
          arc [ radius     =\pgfkeysvalueof{/tikz/kinky cross radius},
                start angle=\n{@kc@},
                delta angle=\pgfkeysvalueof{/tikz/@kinky cross}180 ]
          -- (\tikztotarget)}}
      }

% Front matter, main matter, and back matter definitions
\def\frontmatter{%
 \pagenumbering{roman}
 \setcounter{page}{1}
 \renewcommand{\thesection}{\roman{section}}
}%
\def\mainmatter{%
 \pagenumbering{arabic}
 \setcounter{page}{1}
 \setcounter{section}{0}
 \renewcommand{\thesection}{\arabic{section}}
}%
\def\backmatter{%
 \setcounter{section}{0}
 \renewcommand{\thesection}{\alph{section}}
}%

% Package Configurations
% Allow hyperlinks to be broken
\urlstyle{tt}

% SI unit pkg conf
\sisetup{
    load-configurations = abbreviations,
    detect-family = true,
    per-mode = reciprocal
}%

% Colorx package adding colours
\definecolor{mygreen}{rgb}{0,0.6,0}
\definecolor{mygray}{rgb}{0.5,0.5,0.5}
\definecolor{mymauve}{rgb}{0.58,0,0.82}

% Listings package defining code highlighting colours
\lstset{%
    backgroundcolor=\color{white}, % choose the background color
    breaklines=true, % automatic line breaking only at whitespace
    captionpos=b, % sets the caption-position to bottom
    commentstyle=\color{mygreen}, % comment style
    escapeinside={\%*}{*)}, % if you want to add LaTeX within your code
    keywordstyle=\color{blue}, % keyword style
    stringstyle=\color{mymauve}, % string literal style
    tabsize=2, % Set tabs to be displayed as 2 spaces
    showstringspaces=false, % Don't show string spaces as special characters
    basicstyle=\linespread{1}\footnotesize, % size of fonts used for the code and spacing on the lines (force to 1)
    literate={\ \ }{{\ }}1, % Convert two spaces to 1 (for soft tab)
    numbers=left, % display linenubers to the left
    caption={\protect\filename@parse{\lstname}\protect\filename@base\text{.}\protect\filename@ext} % give filename as caption
}

% Bibliography .bib file locations should maybe be local?
\addbibresource[location=local]{\sharedPath/bibFiles/accelerators.bib}

% Other Configurations

% Setup for having multiline cells
\renewcommand\theadalign{bc}
\renewcommand\theadfont{\bfseries}
\renewcommand\theadgape{\Gape[4pt]}
\renewcommand\cellgape{\Gape[4pt]}
% To add line separating subfigures and to have roman numerals instead of letters
\newcommand{\rulesep}{\unskip\ \vrule\ \vrule\ }
\renewcommand{\thesubfigure}{\alph{subfigure}}
% Dedication environment
\newenvironment{dedication}{
    \clearpage
    \thispagestyle{empty}
    \vspace*{\stretch{1}}
    \itshape
    \raggedleft
}
{
    \par
    \vspace{\stretch{3}}
    \clearpage
}
% Fancyhdr modification of page numbering
\pagestyle{fancy}
\fancyhf{}
\fancyfoot[R]{\thepage}
\renewcommand{\headrulewidth}{0pt}
\renewcommand{\footrulewidth}{1pt}
% Separate list of appendicies
\newcommand{\listappendixname}{List of Appendices}
\newlistof{appendix}{app}{\listappendixname}
\setcounter{appdepth}{3}    
\renewcommand{\theappendix}{\Alph{appendix}}
\renewcommand{\cftappendixpresnum}{Appendix\space}
\setlength{\cftappendixindent}{1.5em}
\setlength{\cftappendixnumwidth}{1in}
\newlistentry[appendix]{subappendix}{app}{1}
\renewcommand{\thesubappendix}{\theappendix.\arabic{subappendix}}
\renewcommand{\cftsubappendixpresnum}{}
\setlength{\cftsubappendixnumwidth}{2.3em}
\setlength{\cftsubappendixindent}{3.0em}
\newlistentry[subappendix]{subsubappendix}{app}{1}
\renewcommand{\thesubsubappendix}{\thesubappendix.\arabic{subsubappendix}}
\renewcommand{\cftsubsubappendixpresnum}{}
\setlength{\cftsubsubappendixnumwidth}{3.2em}
\setlength{\cftsubsubappendixindent}{5.3em}
\newcommand{\appndx}[1]{%
  \refstepcounter{appendix}%
  \section*{\theappendix\space #1}%
  \addcontentsline{app}{appendix}{\protect\numberline{\theappendix}#1}%
  \par
}
\newcommand{\subappndx}[1]{%
  \refstepcounter{subappendix}%
  \subsection*{\thesubappendix\space #1}%
  \addcontentsline{app}{subappendix}{\protect\numberline{\thesubappendix}#1}%
}
\newcommand{\subsubappndx}[1]{%
  \refstepcounter{subsubappendix}%
  \subsection*{\thesubsubappendix\space #1}%
  \addcontentsline{app}{subsubappendix}{\protect\numberline{\thesubsubappendix}#1}%
}
% For stopping figures from appearing outside subsections and subsubsections using placeins
\makeatletter
\AtBeginDocument{%
    \expandafter\renewcommand\expandafter\subsection\expandafter{%
        \expandafter\@fb@secFB\subsection
    }%
}
\makeatother
\makeatletter
\AtBeginDocument{%
    \expandafter\renewcommand\expandafter\subsubsection\expandafter{%
        \expandafter\@fb@secFB\subsubsection
    }%
}
\makeatother
% For using the standalone package
\newboolean{standaloneFlag}
\setboolean{standaloneFlag}{true}
% Add Constants list using glossary
\newglossary[cgls]{constants}{cstog}{cstig}{Constants}
% A new command for less cramped nested fractions
\newcommand\ddfrac[2]{\frac{\displaystyle #1}{\displaystyle #2}}
% Increasing space in matrices
\makeatletter
\renewcommand*\env@matrix[1][\arraystretch]{%
    \edef\arraystretch{#1}%
    \hskip -\arraycolsep
    \let\@ifnextchar\new@ifnextchar
    \array{*\c@MaxMatrixCols c}
}
\makeatother
% Alphabetize glossary and acronyms list
\makeglossaries

%\pathtoluacode{\codePath/boteCalc.lua}
%\yamltolua{\assetPath/Data/boteVars.yaml}{fmtVar}
%\makeLuaGloss{fmtVar}
% Command to conditionally typeset a bibliography.
\newcommand{\standaloneBib}{%%
  \ifthenelse{\boolean{standaloneFlag}}%%
  {\printbibliography[heading=bibintoc]
        \printglossary[type=symbols]
        \printglossary[type=acronymtype]
  \printglossary[type=main]}{}}

%%%%%%%%%%%%%%%%%%%
%%%%%%%%%%%%%%%% Acronym Only
\newglossaryentry{ndfeb}
{
    type=\acronymtype,
    name={NdFeB},
    description={Neodymium Iron Boron Rare Earth Permanent Magnet},
    first={Neodymium iron boron (NdFeB)}
}
\newglossaryentry{smco}
{
    type=\acronymtype,
    name={SmCo},
    description={Samarium Cobalt Rare Earth Permanent Magnet},
    first={Samarium cobalt (SmCo)}
}
\newglossaryentry{cnc}
{
    type=\acronymtype,
    name={CNC},
    description={Computer Numerically Controlled},
    first={Computer Numerically Controlled (CNC)}
}
\newglossaryentry{ptfe}
{
    type=\acronymtype,
    name={PTFE},
    description={Polytetraflouroethylene (PTFE), commonly referred to by the brand name Teflon, is a synthetic polymer with highly desirable electrical breakdown characteristics},
    first={Polytetraflouroethylene (PTFE)}
}

%%%%%%%%%%%%%%%%%%%
%%%%%%%%%%%%%%%% Glossary Only
\newglossaryentry{cpp}
{%
    name={C++},
    description={C++ is a programming language that can be used as an object oriented programming language, an imperative programming language, and still provide low-level memory control. Note: All C++ code used in this work is compiled under the C++11 standard}
}
\newglossaryentry{python}
{%
    name={Python},
    description={Python is a flexible scripting language that can be used as an imperative language, an object oriented language. Note: All python code used in this work is written and executed using Python3.6}
}
\newglossaryentry{ibsimu}
{%
    name={IBSIMU},
    description={IBSIMU is a \gls{cpp} library for ion optics, space-charge calculation, and plasma extraction. It is able to import 2-D and 3-D geometries as exported in .STL or .DXF file formats. It also allows for mathematical definition of geometries. Note: Version 1.0.6 of the IBSIMU library was used for this work \cite{acc:14}}
}
\newglossaryentry{ansys}
{
    name={ANSYS},
    description={ANSYS Multiphysics is a software suite for multiphysics modelling and simulation}
}

%%%%%%%%%%%%%%%%%%%
%%%%%%%%%%%%%%%% Glossary and Acronym
\newglossaryentry{hvag}
{%
 name={HV},
 description={High Vacuum (HV) is a term which refers to vacuum pressures in the range of \SI{1e-3}{\torr} to \SI{1e-8}{\torr}\cite{acc:13}}
}
\newglossaryentry{hva}
{%
 type=\acronymtype,
 name={HV},
 description={High Vacuum},
 first={High Vacuum (HV)\glsadd{hvag}},
 see=[Glossary:]{hvag}
}

%%% Constants
%%%%%%%%%%%%%%%%%%%%%%%%%%%%%%%%%%%%%%%%%%%%%%%%%%%%%%%%%%%%%%%%%%%%%%%%%%%%%%%%%%%%%%%%
\newglossaryentry{c}
{
    type=constants,
    name={\ensuremath{c}},
    description={\mbox{} speed of light (\SI{2.997924e8}{\meter\per\second})}
}
\newglossaryentry{boltz}
{
    type=constants,
    name={\ensuremath{k_{B}}},
    description={\mbox{} Boltzmann constant (\SI{8.617330e-5}{\electronvolt\per\kelvin})}
}
\newglossaryentry{eps0}
{
    type=constants,
    name={\ensuremath{\varepsilon_{0}}},
    description={\mbox{} permittivity of free space (\SI{8.854187e-12}{\farad\per\meter})}
}
\newglossaryentry{eq}
{
    type=constants,
    name={\ensuremath{e}},
    description={\mbox{} elementary charge (\SI{1.602176e-19}{\coulomb})}
}

%%% Symbols
%%%%%%%%%%%%%%%%%%%%%%%%%%%%%%%%%%%%%%%%%%%%%%%%%%%%%%%%%%%%%%%%%%%%%%%%%%%%%%%%%%%%%%%%
\newglossaryentry{mg}
{
    type=symbols,
    name={\ensuremath{G}},
    description={Mosfet Gate}
}
\newglossaryentry{md}
{
    type=symbols,
    name={\ensuremath{D}},
    description={Mosfet Drain}
}
\newglossaryentry{ms}
{
    type=symbols,
    name={\ensuremath{S}},
    description={Mosfet Source}
}
\newglossaryentry{press}
{
    type=symbols,
    name={\ensuremath{P}},
    description={Pressure (units of \si{\torr})}
}
\newglossaryentry{time}
{
    type=symbols,
    name={\ensuremath{t}},
    description={time (units of minutes)}
}
\newglossaryentry{ener}
{
    type=symbols,
    name={\ensuremath{E}},
    description={Energy}
}
\newglossaryentry{efermi}
{
    type=symbols,
    name={\ensuremath{E_{f}}},
    description={Fermi Energy of a Material}
}
\newglossaryentry{fermiDist}
{
    type=symbols,
    name={\ensuremath{f(E)}},
    description={The Fermi-Dirac distribution}
}
\newglossaryentry{econd}
{
    type=symbols,
    name={\ensuremath{E_{C}}},
    description={Conduction band energy level}
}
\newglossaryentry{evalence}
{
    type=symbols,
    name={\ensuremath{E_{V}}},
    description={Valence band energy level}
}
\newglossaryentry{egap}
{
    type=symbols,
    name={\ensuremath{E_{G}}},
    description={Bandgap}
}
\newglossaryentry{rd}
{
    type=symbols,
    name={\ensuremath{R_{D}}},
    description={Resistance}
}
\newglossaryentry{vout}
{
    type=symbols,
    name={\ensuremath{V_{out}}},
    description={Output voltage}
}
\newglossaryentry{vin}
{
    type=symbols,
    name={\ensuremath{V_{in}}},
    description={Input voltage}
}
\newglossaryentry{temp}
{
    type=symbols,
    name={\ensuremath{T}},
    description={Temperature in units of Kelvin}
}
\newglossaryentry{pop}
{
    type=symbols,
    name={\ensuremath{Pop}},
    description={Population}
}

\newglossaryentry{sig}
{
    type=symbols,
    name={\ensuremath{\sigma}},
    description={Lorenz Parameter}
}
\newglossaryentry{rho}
{
    type=symbols,
    name={\ensuremath{\rho}},
    description={Lorenz Parameter}
}
\newglossaryentry{beta}
{
    type=symbols,
    name={\ensuremath{\beta}},
    description={Lorenz Parameter}
}



\begin{document}
    \directlua{
        path = "\plotCodePath"
        gol = require(path.."/Code/gol")
    }
    \newcommand\addLuaGol[4][]{%
        \directlua{
            #4 = gol.golRun(#2,#3,#4,[[#1]])
        }%
    }
    \newcommand\countLuaGol[6][]{%
        \directlua{
            #2 = gol.golTot(#2,#3,#4,#5,#6,[[#1]])
        }%
    }
    \begin{animateinline}[poster=last, controls, buttonsize=1.0em]{12}
        \multiframe{100}{iTer=0+1}{
            \begin{tikzpicture}
                \begin{groupplot}[
                    group style={
                        group size =1 by 2,
                        vertical sep=1.8cm,
                    }
                ]
                \nextgroupplot[grid=both,
                    colormap/blackwhite,
                    xmax=40,
                    xmin=0,
                    ymin=0,
                    ymax=40,
                    width=10cm,
                    height=10cm,
                    xlabel=$X$,
                    ylabel=$Y$,]
                    \addLuaGol{41}{41}{mat};
                \nextgroupplot[grid=both,
                    width=10cm,
                    height=4cm,
                    xmin=0,
                    xmax=100,
                    ymin=600, 
                    ymax=750,
                    xlabel=\gls{time},
                    ylabel=\gls{pop},]
                    \countLuaGol[mark=*, mark size=0.8, blue]{count}{mat}{40}{40}{\iTer};
                \end{groupplot}
            \end{tikzpicture}
        }
    \end{animateinline}
\end{document}

            % Currently there is an issue with tikzscale that prevents this from working with pgfplots next to eachother
            \includestandalone[height=0.7\textheight]{\assetPath/Images/Tikz/gol/gol}
            %\import{\assetPath/Images/Tikz/gol/}{gol.tikz}
            %\includestandalone[width=\textwidth]{\assetPath/Images/Tikz/gol/gol}
            \caption{Game of life running in lua and plotting in tikz}
            \label{fig:animeGol}
        \end{centering}
    \end{figure}
%</golAnimate>
%<*golAnimatePres>
    \begin{figure}[htbp]
        \begin{centering}
            %% arara: lualatex: { shell: true}
% arara: lualatex: { shell: true}

\providecommand{\importPath}{../../../../Shared/Imports}
\providecommand{\assetPath}{../../../../Assets}
\providecommand{\codePath}{../../../../Shared/LuaCalcs}
\providecommand{\sharedPath}{../../../../Shared}
\providecommand{\plotCodePath}{../../../../Assets}

\documentclass[tikz, crop=false, export]{standalone}

\input{\codePath/luaDatSetup}
\usepackage[mode=buildnew, subpreambles=false]{standalone}
\usepackage{import}
%\usepackage[a-1b]{pdfx}
\usepackage{indentfirst}
\usepackage[doublespacing]{setspace}
\usepackage[left=1.5in, right=1in, top=1in, bottom=1in]{geometry}
\usepackage{xurl}
\usepackage{hyperref}
\usepackage[tbtags]{amsmath}
\usepackage{amsfonts}
\usepackage{amssymb}
\usepackage{fancyhdr}
\usepackage[T1]{fontenc}
\usepackage{comment}
\usepackage{xparse}
\usepackage{xcolor}
\usepackage{colortbl}
\usepackage{siunitx}
\usepackage{chemformula}
\usepackage{ifthen}
\usepackage[gobble=auto]{pythontex}
\usepackage{catchfilebetweentags}
\usepackage[style=ieee,backend=biber]{biblatex}
\usepackage{csvsimple}
\usepackage{longtable}
\usepackage{makecell}
\usepackage{multicol}
\usepackage{multirow}
\usepackage{float}
\usepackage{enumitem}
\usepackage{graphicx}

\usepackage{tikz}
    \usetikzlibrary{math, arrows, circuits.ee.IEC, positioning, shapes.arrows, shapes.geometric, external}
\usepackage{pgfplots}
    \pgfplotsset{compat=newest, compat/show suggested version=false}
    \usepgfplotslibrary{groupplots}
\usepackage[siunitx,american voltages, american currents, RPvoltages]{circuitikz}
\usepackage{tikzscale}

\usepackage{animate}
\usepackage{caption}
\usepackage{subcaption}
\usepackage{listings}
\usepackage{chngcntr}
\usepackage[titletoc]{appendix}
\usepackage[titles]{tocloft}
\usepackage{xpatch}
\usepackage[debug, toc, section=section, acronym, symbols]{glossaries} % Glossaries package

%Get rounded wire jumps
\tikzset{
    declare function={% in case of CVS which switches the arguments of atan2
        atan3(\a,\b)=ifthenelse(atan2(0,1)==90, atan2(\a,\b), atan2(\b,\a));},
        kinky cross radius/.initial=+.125cm,
        @kinky cross/.initial=+, kinky crosses/.is choice,
        kinky crosses/left/.style={@kinky cross=-},kinky crosses/right/.style={@kinky cross=+},
        kinky cross/.style args={(#1)--(#2)}{
        to path={
          let \p{@kc@}=($(\tikztotarget)-(\tikztostart)$),
              \n{@kc@}={atan3(\p{@kc@})+180} in
          -- ($(intersection of \tikztostart--{\tikztotarget} and #1--#2)!%
                 \pgfkeysvalueof{/tikz/kinky cross radius}!(\tikztostart)$)
          arc [ radius     =\pgfkeysvalueof{/tikz/kinky cross radius},
                start angle=\n{@kc@},
                delta angle=\pgfkeysvalueof{/tikz/@kinky cross}180 ]
          -- (\tikztotarget)}}
      }

% Front matter, main matter, and back matter definitions
\def\frontmatter{%
 \pagenumbering{roman}
 \setcounter{page}{1}
 \renewcommand{\thesection}{\roman{section}}
}%
\def\mainmatter{%
 \pagenumbering{arabic}
 \setcounter{page}{1}
 \setcounter{section}{0}
 \renewcommand{\thesection}{\arabic{section}}
}%
\def\backmatter{%
 \setcounter{section}{0}
 \renewcommand{\thesection}{\alph{section}}
}%

% Package Configurations
% Allow hyperlinks to be broken
\urlstyle{tt}

% SI unit pkg conf
\sisetup{
    load-configurations = abbreviations,
    detect-family = true,
    per-mode = reciprocal
}%

% Colorx package adding colours
\definecolor{mygreen}{rgb}{0,0.6,0}
\definecolor{mygray}{rgb}{0.5,0.5,0.5}
\definecolor{mymauve}{rgb}{0.58,0,0.82}

% Listings package defining code highlighting colours
\lstset{%
    backgroundcolor=\color{white}, % choose the background color
    breaklines=true, % automatic line breaking only at whitespace
    captionpos=b, % sets the caption-position to bottom
    commentstyle=\color{mygreen}, % comment style
    escapeinside={\%*}{*)}, % if you want to add LaTeX within your code
    keywordstyle=\color{blue}, % keyword style
    stringstyle=\color{mymauve}, % string literal style
    tabsize=2, % Set tabs to be displayed as 2 spaces
    showstringspaces=false, % Don't show string spaces as special characters
    basicstyle=\linespread{1}\footnotesize, % size of fonts used for the code and spacing on the lines (force to 1)
    literate={\ \ }{{\ }}1, % Convert two spaces to 1 (for soft tab)
    numbers=left, % display linenubers to the left
    caption={\protect\filename@parse{\lstname}\protect\filename@base\text{.}\protect\filename@ext} % give filename as caption
}

% Bibliography .bib file locations should maybe be local?
\addbibresource[location=local]{\sharedPath/bibFiles/accelerators.bib}

% Other Configurations

% Setup for having multiline cells
\renewcommand\theadalign{bc}
\renewcommand\theadfont{\bfseries}
\renewcommand\theadgape{\Gape[4pt]}
\renewcommand\cellgape{\Gape[4pt]}
% To add line separating subfigures and to have roman numerals instead of letters
\newcommand{\rulesep}{\unskip\ \vrule\ \vrule\ }
\renewcommand{\thesubfigure}{\alph{subfigure}}
% Dedication environment
\newenvironment{dedication}{
    \clearpage
    \thispagestyle{empty}
    \vspace*{\stretch{1}}
    \itshape
    \raggedleft
}
{
    \par
    \vspace{\stretch{3}}
    \clearpage
}
% Fancyhdr modification of page numbering
\pagestyle{fancy}
\fancyhf{}
\fancyfoot[R]{\thepage}
\renewcommand{\headrulewidth}{0pt}
\renewcommand{\footrulewidth}{1pt}
% Separate list of appendicies
\newcommand{\listappendixname}{List of Appendices}
\newlistof{appendix}{app}{\listappendixname}
\setcounter{appdepth}{3}    
\renewcommand{\theappendix}{\Alph{appendix}}
\renewcommand{\cftappendixpresnum}{Appendix\space}
\setlength{\cftappendixindent}{1.5em}
\setlength{\cftappendixnumwidth}{1in}
\newlistentry[appendix]{subappendix}{app}{1}
\renewcommand{\thesubappendix}{\theappendix.\arabic{subappendix}}
\renewcommand{\cftsubappendixpresnum}{}
\setlength{\cftsubappendixnumwidth}{2.3em}
\setlength{\cftsubappendixindent}{3.0em}
\newlistentry[subappendix]{subsubappendix}{app}{1}
\renewcommand{\thesubsubappendix}{\thesubappendix.\arabic{subsubappendix}}
\renewcommand{\cftsubsubappendixpresnum}{}
\setlength{\cftsubsubappendixnumwidth}{3.2em}
\setlength{\cftsubsubappendixindent}{5.3em}
\newcommand{\appndx}[1]{%
  \refstepcounter{appendix}%
  \section*{\theappendix\space #1}%
  \addcontentsline{app}{appendix}{\protect\numberline{\theappendix}#1}%
  \par
}
\newcommand{\subappndx}[1]{%
  \refstepcounter{subappendix}%
  \subsection*{\thesubappendix\space #1}%
  \addcontentsline{app}{subappendix}{\protect\numberline{\thesubappendix}#1}%
}
\newcommand{\subsubappndx}[1]{%
  \refstepcounter{subsubappendix}%
  \subsection*{\thesubsubappendix\space #1}%
  \addcontentsline{app}{subsubappendix}{\protect\numberline{\thesubsubappendix}#1}%
}
% For stopping figures from appearing outside subsections and subsubsections using placeins
\makeatletter
\AtBeginDocument{%
    \expandafter\renewcommand\expandafter\subsection\expandafter{%
        \expandafter\@fb@secFB\subsection
    }%
}
\makeatother
\makeatletter
\AtBeginDocument{%
    \expandafter\renewcommand\expandafter\subsubsection\expandafter{%
        \expandafter\@fb@secFB\subsubsection
    }%
}
\makeatother
% For using the standalone package
\newboolean{standaloneFlag}
\setboolean{standaloneFlag}{true}
% Add Constants list using glossary
\newglossary[cgls]{constants}{cstog}{cstig}{Constants}
% A new command for less cramped nested fractions
\newcommand\ddfrac[2]{\frac{\displaystyle #1}{\displaystyle #2}}
% Increasing space in matrices
\makeatletter
\renewcommand*\env@matrix[1][\arraystretch]{%
    \edef\arraystretch{#1}%
    \hskip -\arraycolsep
    \let\@ifnextchar\new@ifnextchar
    \array{*\c@MaxMatrixCols c}
}
\makeatother
% Alphabetize glossary and acronyms list
\makeglossaries

%\pathtoluacode{\codePath/boteCalc.lua}
%\yamltolua{\assetPath/Data/boteVars.yaml}{fmtVar}
%\makeLuaGloss{fmtVar}
% Command to conditionally typeset a bibliography.
\newcommand{\standaloneBib}{%%
  \ifthenelse{\boolean{standaloneFlag}}%%
  {\printbibliography[heading=bibintoc]
        \printglossary[type=symbols]
        \printglossary[type=acronymtype]
  \printglossary[type=main]}{}}

%%%%%%%%%%%%%%%%%%%
%%%%%%%%%%%%%%%% Acronym Only
\newglossaryentry{ndfeb}
{
    type=\acronymtype,
    name={NdFeB},
    description={Neodymium Iron Boron Rare Earth Permanent Magnet},
    first={Neodymium iron boron (NdFeB)}
}
\newglossaryentry{smco}
{
    type=\acronymtype,
    name={SmCo},
    description={Samarium Cobalt Rare Earth Permanent Magnet},
    first={Samarium cobalt (SmCo)}
}
\newglossaryentry{cnc}
{
    type=\acronymtype,
    name={CNC},
    description={Computer Numerically Controlled},
    first={Computer Numerically Controlled (CNC)}
}
\newglossaryentry{ptfe}
{
    type=\acronymtype,
    name={PTFE},
    description={Polytetraflouroethylene (PTFE), commonly referred to by the brand name Teflon, is a synthetic polymer with highly desirable electrical breakdown characteristics},
    first={Polytetraflouroethylene (PTFE)}
}

%%%%%%%%%%%%%%%%%%%
%%%%%%%%%%%%%%%% Glossary Only
\newglossaryentry{cpp}
{%
    name={C++},
    description={C++ is a programming language that can be used as an object oriented programming language, an imperative programming language, and still provide low-level memory control. Note: All C++ code used in this work is compiled under the C++11 standard}
}
\newglossaryentry{python}
{%
    name={Python},
    description={Python is a flexible scripting language that can be used as an imperative language, an object oriented language. Note: All python code used in this work is written and executed using Python3.6}
}
\newglossaryentry{ibsimu}
{%
    name={IBSIMU},
    description={IBSIMU is a \gls{cpp} library for ion optics, space-charge calculation, and plasma extraction. It is able to import 2-D and 3-D geometries as exported in .STL or .DXF file formats. It also allows for mathematical definition of geometries. Note: Version 1.0.6 of the IBSIMU library was used for this work \cite{acc:14}}
}
\newglossaryentry{ansys}
{
    name={ANSYS},
    description={ANSYS Multiphysics is a software suite for multiphysics modelling and simulation}
}

%%%%%%%%%%%%%%%%%%%
%%%%%%%%%%%%%%%% Glossary and Acronym
\newglossaryentry{hvag}
{%
 name={HV},
 description={High Vacuum (HV) is a term which refers to vacuum pressures in the range of \SI{1e-3}{\torr} to \SI{1e-8}{\torr}\cite{acc:13}}
}
\newglossaryentry{hva}
{%
 type=\acronymtype,
 name={HV},
 description={High Vacuum},
 first={High Vacuum (HV)\glsadd{hvag}},
 see=[Glossary:]{hvag}
}

%%% Constants
%%%%%%%%%%%%%%%%%%%%%%%%%%%%%%%%%%%%%%%%%%%%%%%%%%%%%%%%%%%%%%%%%%%%%%%%%%%%%%%%%%%%%%%%
\newglossaryentry{c}
{
    type=constants,
    name={\ensuremath{c}},
    description={\mbox{} speed of light (\SI{2.997924e8}{\meter\per\second})}
}
\newglossaryentry{boltz}
{
    type=constants,
    name={\ensuremath{k_{B}}},
    description={\mbox{} Boltzmann constant (\SI{8.617330e-5}{\electronvolt\per\kelvin})}
}
\newglossaryentry{eps0}
{
    type=constants,
    name={\ensuremath{\varepsilon_{0}}},
    description={\mbox{} permittivity of free space (\SI{8.854187e-12}{\farad\per\meter})}
}
\newglossaryentry{eq}
{
    type=constants,
    name={\ensuremath{e}},
    description={\mbox{} elementary charge (\SI{1.602176e-19}{\coulomb})}
}

%%% Symbols
%%%%%%%%%%%%%%%%%%%%%%%%%%%%%%%%%%%%%%%%%%%%%%%%%%%%%%%%%%%%%%%%%%%%%%%%%%%%%%%%%%%%%%%%
\newglossaryentry{mg}
{
    type=symbols,
    name={\ensuremath{G}},
    description={Mosfet Gate}
}
\newglossaryentry{md}
{
    type=symbols,
    name={\ensuremath{D}},
    description={Mosfet Drain}
}
\newglossaryentry{ms}
{
    type=symbols,
    name={\ensuremath{S}},
    description={Mosfet Source}
}
\newglossaryentry{press}
{
    type=symbols,
    name={\ensuremath{P}},
    description={Pressure (units of \si{\torr})}
}
\newglossaryentry{time}
{
    type=symbols,
    name={\ensuremath{t}},
    description={time (units of minutes)}
}
\newglossaryentry{ener}
{
    type=symbols,
    name={\ensuremath{E}},
    description={Energy}
}
\newglossaryentry{efermi}
{
    type=symbols,
    name={\ensuremath{E_{f}}},
    description={Fermi Energy of a Material}
}
\newglossaryentry{fermiDist}
{
    type=symbols,
    name={\ensuremath{f(E)}},
    description={The Fermi-Dirac distribution}
}
\newglossaryentry{econd}
{
    type=symbols,
    name={\ensuremath{E_{C}}},
    description={Conduction band energy level}
}
\newglossaryentry{evalence}
{
    type=symbols,
    name={\ensuremath{E_{V}}},
    description={Valence band energy level}
}
\newglossaryentry{egap}
{
    type=symbols,
    name={\ensuremath{E_{G}}},
    description={Bandgap}
}
\newglossaryentry{rd}
{
    type=symbols,
    name={\ensuremath{R_{D}}},
    description={Resistance}
}
\newglossaryentry{vout}
{
    type=symbols,
    name={\ensuremath{V_{out}}},
    description={Output voltage}
}
\newglossaryentry{vin}
{
    type=symbols,
    name={\ensuremath{V_{in}}},
    description={Input voltage}
}
\newglossaryentry{temp}
{
    type=symbols,
    name={\ensuremath{T}},
    description={Temperature in units of Kelvin}
}
\newglossaryentry{pop}
{
    type=symbols,
    name={\ensuremath{Pop}},
    description={Population}
}

\newglossaryentry{sig}
{
    type=symbols,
    name={\ensuremath{\sigma}},
    description={Lorenz Parameter}
}
\newglossaryentry{rho}
{
    type=symbols,
    name={\ensuremath{\rho}},
    description={Lorenz Parameter}
}
\newglossaryentry{beta}
{
    type=symbols,
    name={\ensuremath{\beta}},
    description={Lorenz Parameter}
}



\begin{document}
    \directlua{
        path = "\plotCodePath"
        gol = require(path.."/Code/gol")
    }
    \newcommand\addLuaGol[4][]{%
        \directlua{
            #4 = gol.golRun(#2,#3,#4,[[#1]])
        }%
    }
    \newcommand\countLuaGol[6][]{%
        \directlua{
            #2 = gol.golTot(#2,#3,#4,#5,#6,[[#1]])
        }%
    }
    \begin{animateinline}[poster=last, controls, buttonsize=1.0em]{12}
        \multiframe{100}{iTer=0+1}{
            \begin{tikzpicture}
                \begin{groupplot}[
                    group style={
                        group size =1 by 2,
                        vertical sep=1.8cm,
                    }
                ]
                \nextgroupplot[grid=both,
                    colormap/blackwhite,
                    xmax=40,
                    xmin=0,
                    ymin=0,
                    ymax=40,
                    width=10cm,
                    height=10cm,
                    xlabel=$X$,
                    ylabel=$Y$,]
                    \addLuaGol{41}{41}{mat};
                \nextgroupplot[grid=both,
                    width=10cm,
                    height=4cm,
                    xmin=0,
                    xmax=100,
                    ymin=600, 
                    ymax=750,
                    xlabel=\gls{time},
                    ylabel=\gls{pop},]
                    \countLuaGol[mark=*, mark size=0.8, blue]{count}{mat}{40}{40}{\iTer};
                \end{groupplot}
            \end{tikzpicture}
        }
    \end{animateinline}
\end{document}

            % Currently there is an issue with tikzscale that prevents this from working with pgfplots next to eachother
            \includestandalone[height=0.7\textheight]{\assetPath/Images/Tikz/gol/gol}
            %\import{\assetPath/Images/Tikz/gol/}{gol.tikz}
            %\includestandalone[width=\textwidth]{\assetPath/Images/Tikz/gol/gol}
            \caption{Game of life running in lua and plotting in tikz}
            \label{fig:animeGol}
        \end{centering}
    \end{figure}
%</golAnimatePres>

%%% Image Series
%%%%%%%%%%%%%%%%%%%%%%%%%%%%%%%%%%%%%%%%%%%%%%%%%%%%%%%%%%%%%%%%%%%%%%%%%%%%%%%%%%%%%%%%
%<*processImgs>
    \begin{figure}[htbp]
        \begin{centering}
            \begin{overlayarea}{\textwidth}{\textheight}
                \only<1>{\includegraphics[width=\textwidth]{example-image-a}}%
                \only<2>{\includegraphics[width=\textwidth]{example-image-b}}%
                \only<3>{\includegraphics[width=\textwidth]{example-image-c}}%
            \end{overlayarea}
        \end{centering}
    \end{figure}
%</processImgs>

%%% Tables
%%%%%%%%%%%%%%%%%%%%%%%%%%%%%%%%%%%%%%%%%%%%%%%%%%%%%%%%%%%%%%%%%%%%%%%%%%%%%%%%%%%%%%%%
%<*targsAndVals>
    \begin{table}[H]
        \centering
        \caption{Table of specified parameters and achieved values (a green cell indicates a "Pass" and a red cell a "Fail")}
        \begin{tabular}{c|c|c|c|c|c}
            \bfseries \thead{Parameter\\($f_T\ 6.881\ GHz$)\\($f_{if}\ 200\ MHz$)} & \bfseries \thead{System\\Target} & \bfseries \thead{Theoretical\\Values\\(Mixer Only)} & \bfseries \thead{Initial\\Results\\(Mixer Only)} & \bfseries \thead{Initial\\Results\\System} & \bfseries \thead{Final\\System\\Design}
            \csvreader[head to column names]{\assetPath/Data/TargetsAndVals.csv}{}
            {\\\hline\csvcoli & \csvcolii & \csvcoliii & \csvcoliv & \csvcolv & \csvcolvi}
        \end{tabular}
        \label{tab:Parameters}
    \end{table}
%</targsAndVals>

%%% Tikz Figures
%%%%%%%%%%%%%%%%%%%%%%%%%%%%%%%%%%%%%%%%%%%%%%%%%%%%%%%%%%%%%%%%%%%%%%%%%%%%%%%%%%%%%%%%
%<*tikzplot>
    \begin{figure}[htbp]
        \begin{centering}
            \includestandalone[width=0.75\textwidth]{\assetPath/Images/Tikz/plot/plot}
            \caption{Lorenz Double Scroll Produced in LuaLatex}
            \label{fig:plottikz}
        \end{centering}
    \end{figure}
%</tikzplot>

%<*tikzLorenz>
    \begin{figure}[htbp]
        \begin{centering}
            \includestandalone[width=0.85\textwidth]{\assetPath/Images/Tikz/lorenz/lorenzTikz}
            \caption{Lorenz Double Scroll Produced in LuaLatex}
            \label{fig:lorScr}
        \end{centering}
    \end{figure}
%</tikzLorenz>

%<*tikzCircuit>
    \begin{figure}[htbp]
        \begin{centering}
            \includestandalone[width=1\textwidth]{\assetPath/Images/Tikz/Circuits/circuit}
            \caption{A Tikz Circuit Diagram}
            \label{fig:circTikz}
        \end{centering}
    \end{figure}
%</tikzCircuit>

%<*tikzDiag>
    \begin{figure}[htbp]
        \begin{centering}
            \includestandalone[width=\textwidth]{\assetPath/Images/Tikz/diagram/diag}
            \caption{A Tikz sketch}
            \label{fig:diagTiks}
        \end{centering}
    \end{figure}
%</tikzDiag>

%%% Figures
%%%%%%%%%%%%%%%%%%%%%%%%%%%%%%%%%%%%%%%%%%%%%%%%%%%%%%%%%%%%%%%%%%%%%%%%%%%%%%%%%%%%%%%%
%<*completedChamber>
    \begin{figure}[tbph]
        \begin{centering}
            \includegraphics[width=0.9\textwidth]{example-image-G}
            \caption{Completed vacuum chamber and pump, sealed for initial vacuum testing}
            \label{fig:completedChamber}
        \end{centering}
    \end{figure}
%</completedChamber>
%<*mountPIGEinz>
    \begin{figure}[tbph]
        \begin{centering}
            \includegraphics[width=\textwidth]{example-image-H}
            \caption{Mounted, electrically connected, and gas connected \gls{pig} with Einzel lens}
            \label{fig:mountPIGEinz}
        \end{centering}
    \end{figure}%
%</mountPIGEinz>
